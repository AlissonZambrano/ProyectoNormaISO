\documentclass[12pt,a4paper]{article}
\usepackage[utf8]{inputenc}
\usepackage[spanish]{babel}
\usepackage{amsmath}
\usepackage{amsfonts}
\usepackage{amssymb}
\usepackage{makeidx}
\usepackage{graphicx}
\usepackage[left=2cm,right=2cm,top=2cm,bottom=2cm]{geometry}

\title{PROYECTO}
\author{Chiriguaya Alvaro, Mite Lady, Zambrano Allison}
\date{2 de marzo del 2020}
\begin{document}
\maketitle

\section{Normas ISO 9001}
La ISO 9001 (International Standarization Organization) es una norma internacional que toma en cuenta las actividades de una organización, sin distinción de sector de actividad. Esta norma se concentra en la satisfacción del cliente y en la capacidad de proveer productos y servicios que cumplan con las exigencias internas y externas de la organización. Hoy por hoy, la norma ISO 9001 es la norma de mayor renombre y la más utilizada alrededor del mundo.

\section{Caso de estudio}
El TPV o POS (Point of Sale) es la evolución del siglo XXI de la clásica caja registradora. En el mundo de la hostelería y venta al por menor de pequeños comercios los TPV son utilizados para registrar ventas, beneficios, pedidos, inventarios, historial de clientes, etc. En resumen, un TPV otorga control sobre las operaciones de negocio.\\
Un TPV básico consiste en una computadora, un cajón de dinero, una impresora de tiques, un monitor y dispositivos de lectura como lector de códigos de barras, teclados, etc. Los TPV registran las transacciones y permiten generar detallados informes, permitiendo tomar mejores decisiones comerciales. El TPV correcto permitirá mejorar la productividad y redundará en mejores ganancias.\\
Un TPV puede revolucionar el funcionamiento de un comercio. Desde incrementar la productividad de los empleados hasta conocer si las inversiones obtienen beneficios.\\\\
\begin{center}
\includegraphics[width=0.6\textwidth]{../../Downloads/01.jpg} 
\end{center}

\section{Descripción del sistema de FastFood}
Realización del producto.\\\\
El sistema de calidad según la ISO 9001 y desean implantar un sistema de gestión integrado, deben realizar una planificación y desarrollo de los procesos implicados en la realización de productos estos e refiere a su funcionalidad.
\begin{itemize}
\item Proveedores
\item Productos
\item Familias
\item Empleados
\item Clientes
\item Caja
\item Stock
\item Configuración
\item Listados
\item Acerca de (Para mas información)
\item Nivel acceso\\\\
\end{itemize}
\begin{center}
\includegraphics[width=0.7\textwidth]{../../Downloads/02.jpeg} 
\end{center}

Procesos relacionados con el cliente.\\\\
Según la norma ISO 9001, el sistema debe de contar con registros donde se puede ingresar con el usuario de un empleado y poder facturar de forma detallada y especifica los productos, y a su vez ofrecer información sobre ellos conforme se va realizando las compras.
\begin{itemize}
\item Los requisitos especificados por el cliente, incluyendo los requisitos para las actividades de entrega y las posteriores a las mismas.
\item Los requisitos no establecidos por el cliente (los productos) pero necesarios para el uso especificado o para el uso previsto, cuando sea conocido y;
\item Cualquier requisito adicional determinado por la organización.
\item Los requisitos legales y reglamentarios relacionados con el producto.\\\\
\end{itemize}
\begin{center}
\includegraphics[width=0.7\textwidth]{../../Downloads/03.jpeg}  
\end{center}

Comunicación con el cliente.\\\\
Implementar disposiciones para la realización de una buena comunicación con  los clientes.
\begin{itemize}
\item La información sobre el producto,
\item Las consultas, atención de pedidos, modificaciones de pedidos.
\item Buena atención al cliente, incluyendo sus quejas.\\\\
\end{itemize}
\begin{center}
\includegraphics[width=0.7\textwidth]{../../Downloads/04.jpeg} 
\end{center}

Diseño y desarrollo\\\\
Abordamos la interpretación y aplicación de los requisitos del diseño y desarrollo de productos y servicios según la norma ISO 9001  
Nuestro programa incluye la materializacion de resultados de un diseño aprobado asi logrando la atracción y desrrollo de nuevos productos y servicios. El diseño y desarrollo implica crear o determinar como debe ser algo que satisface los requisitos. El diseño y desarrollo se basa en las necesidades del cliente, todos los requisitos deben llevar a cabo. El sistema cuenta con un un panel para la selección del empleado, cliente que va a solicitar el producto y asegura a cierta información de la empresa.\\\\
Validación del diseño y desarrollo\\\\
En la fase de validación del diseño y desarrollo e las normas ISO 9001, se  debe ralizar una revisión como está avanzando el diseño de nuestro producto o servicio. Verificamos para ver si efectivamente lo que hemos estado haciendo cumple con los elementos de entrada. Las verificaciones se establecen en la planificación del proyecto y la realizamos cuando podemos comparar una variable de nuestro diseño con los elementos de entrada. Validamos el funcionamiento de nuestro diseño. Dicho de otra forma, nos aseguramos si el producto o servicio sirve o no para cumplir su cometido.
\end{document}